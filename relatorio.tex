%%%%%%%%%%%%%%%%%%%%%%%%%%%%%%%%%%%%%%%%%%%%%%%%%%%%%%%%%%%%%%%%%%%%%%
% How to use writeLaTeX: 
%
% You edit the source code here on the left, and the preview on the
% right shows you the result within a few seconds.
%
% Bookmark this page and share the URL with your co-authors. They can
% edit at the same time!
%
% You can upload figures, bibliographies, custom classes and
% styles using the files menu.
%
%%%%%%%%%%%%%%%%%%%%%%%%%%%%%%%%%%%%%%%%%%%%%%%%%%%%%%%%%%%%%%%%%%%%%%

\documentclass[12pt]{article}

\usepackage{sbc-template}

\usepackage{graphicx,url}

\usepackage{enumitem}

\usepackage{tabularx}

%\usepackage[brazil]{babel}   
\usepackage[utf8]{inputenc}  

     
\sloppy

\title{Trabalho Prático\\ Biblioteca para Manipulação de Grafos}

\author{Stephanie M. Cardozo, Rafael Silva Oliveira}


\address{Pontifícia Universidade Católica de Minas Gerais (PUC Minas)\\
  Código Postal 32.010-025 -- Contagem -- MG -- Brasil
  \email{scardozo@sga.pucminas.br}
  \email{rafael.oliveira.1329416@sga.pucminas.br}
}

\begin{document} 

\maketitle

\begin{resumo} 
  Este relatório descreve as funções e os métodos de uma biblioteca desenvolvida para manipular grafos e representá-los em listas de adjacência, arquivos GML e matrizes de adjacência. A biblioteca utiliza o algoritmo de Fleury para mostrar um caminho euleriano e usa os princípios de Naive/Tarjan para identificar pontes. Além de fornecer opções para criar grafos e realizar consultas, o código fonte também é capaz de ler arquivos GML.
\end{resumo}


\section{Introdução} \label{sec:firstpage}

O objetivo do trabalho é realizar estudos sobre grafos, entender o funcionamento de determinados algoritmos e realizar a contrução e a alteração de estruturas. O trabalho está dividido em duas entregas.

Na primeira entrega, o usuário cria um grafo e consegue realizar a inclusão de arestas, consultar o número de vértices, ponderar uma aresta, dentre outros tipos de manipulação. Em seguida, é possível gerar uma representação do grafo.

Caso o usuário tenha optado por gerar um arquivo GML na parte 1, é possível usá-lo como entrada na parte 2 do trabalho e encontrar um caminho euleriano (caso o grafo seja, de fato, euleriano). Na segunda entrega, os métodos de Naive e Tarjan são usados para encontrar pontes.

\section{Materiais e Métodos}

Linguagem utilizada: Python 3.9.0;\\ 
Biblioteca auxiliar: NetworkX 2.8.8;\\
Software para visualização de grafos: Gephi.

Primeiramente, é preciso realizar a instalação do Python no sistema operacional, em seguida, para instalar a biblioteca no computador, basta abrir o Prompt de Comando e digitar: "pip install networkx[default]".

\section{Resultados e Discussões}
\subsection{Primeira Entrega}

Após compilar o código, o usuário tem a opção de criar um grafo ou de inserir o caminho de um arquivo GML existente.

Caso ele escolha criar um grafo, uma mensagem solicita o número de vértices, em seguida, é perguntado se o grafo é direcionado ou não. Se for não-orientado, o usuário digita "1" e o método "nx.Graph" é chamado. Caso seja direcionado, ele deve digitar "2", assim, chamando "nx.DiGraph". Ambos os métodos fazem parte da biblioteca NetworkX e são usados para criar um grafo "G" (objeto que possui dicionários para armazenar vértices, arestas e atributos).

Posteriormente, aparecerá no terminal uma mensagem perguntando o rótulo e o peso do primeiro vértice. A mensagem é repetida para cada um dos nós.

Feito isso, um grafo vazio é criado e o método "menu" é chamado. Se o usuário escolheu inserir o caminho de um arquivo, o menu será chamado logo após.

No menu principal, existem várias opções para manipular o grafo G. Cada opção será detalhada a seguir:

\begin{enumerate}[itemsep=8pt,parsep=8pt]
    \item Criar arestas:\\ Adiciona uma aresta ao grafo G;\\ Parâmetros: vértice de origem, vértice de destino e valor da aresta;\\ Caso o peso seja igual a 0, a aresta será desconsiderada pelo Gephi e pela matriz de adjacência.
    
    \item Remover arestas: Deleta a aresta inserida pelo usuário.
    
    \item Rotular arestas: Adiociona o atributo "rótulo" à uma aresta existente.
    
    \item Adjacência entre vértices: O usuário insere um vértice e a biblioteca retorna os vizinhos deste nó por meio de um iterador, percorrendo o dict de nós e verificando a existência de arestas entre o vértice escolhido e os demais vértices do grafo.
    
    \item Adjacencia entre arestas: O usuário escolhe uma aresta, em seguida, uma lista com todas as arestas do grafo é gerada. Posteriormente, o programa checa se o primeiro vértice da aresta escolhida existe nas demais arestas do grafo, se for verdade, há adjacencia. Esse mesmo procedimento é repetido para o segundo vértice. Se os dois nós existem ao mesmo tempo, trata-se da mesma aresta inserida pelo usuário, logo, não é adjacente.
    
    \item Existência de arestas: A biblioteca verifica se determinada aresta existe no grafo "G" usando o operador "in". Caso exista, a função retorna True, se não existe, retorna False.
    
    \item Quantidade de vértices e de arestas: A biblioteca retorna um número inteiro que representa o tamanho do dicionário em que os vértices estão armazenados. Esse mesmo princípio é utilizado para retornar a quantidade de arestas.
    
    \item Verificar se o grafo é vazio ou completo: Confere o número de arestas e, se for igual a zero, o grafo é vazio. Para checar se é completo, o programa utiliza a seguinte fórmula: "(V * (V - 1))/2". Em um grafo completo, o total de vértices é igual a V; nesse grafo, um vértice se conecta com todos os outros vértices, menos com ele mesmo (V - 1). Isso será repetido para todos os outros vértices (V * V - 1); dividimos por 2 para não contar o mesmo vértice duas vezes após a multiplicação.
    
    \item Imprimir matriz de adjacência: As linhas e colunas são ordenadas de acordo com os nós que estão armazenados na lista de vértices. O atributo da aresta que contém o valor do "peso" é usado para preencher a matriz.
    
    \item Imprimir lista de adjacência: Ao criar um vértice, a biblioteca, automaticamente, armazena-o em uma lista de adjacência (dict). Esta função imprime a lista no terminal.
    
    \item Gerar arquivo: Gera um arquivo com extenção GEXF, o qual é salvo na mesma pasta onde está localizado o código fonte.
\end{enumerate}

\subsection{Segunda Entrega}
Na segunda parte do trabalho, o desafio era encontrar pontes no grafo, utilizando dois métodos diferentes. A sugestão indicada foi Naive e Tarjan, dois métodos distintos, porém, com o mesmo objetivo.

Com o Naive, analisamos as arestas, uma por uma. Se, ao remover determinada aresta, a quantidade de componentes do grafo aumentar, ela é uma ponte. Para isso, foi feita uma busca em profundidade.

Já a estratégia de Tarjan consiste em verificar, de forma recursiva, cada nó do grafo, em forma de corte de arestas. Para isso, são determinados atributos para cada vértice, como: se já foi visitado, o tempo de descoberta e o nível no vértice. A partir do momento em que o atributo visitado for igual a False, o código entra em recursividade, fazendo com que o vizinho do primeiro nó seja visitado, recebendo mais 1 em tempo e nível.

Após percorrer todos os vértices, se o nó raiz possuir mais de um sucessor, o atributo corte recebe True. Se o nó não for raiz, mas o nó vizinho possuir o atributo low maior ou igual ao atributo "time" do nó raiz, o nó raiz também recebe True no atributo corte.

Com esses dois métodos implementados, o próximo desafio foi encontrar um caminho euleriano com o algoritmo de Fleury. Fleury, a partir do vértice raiz, escolhe uma aresta qualquer e, inicialmente, verifica se aquela aresta é uma ponte, não sendo, aquela aresta é removida e verifica-se novamente a quantidade de graus do vértice seguinte, sendo acima de um, faz a verificação de existência de ponte, até percorrer todos os vertices. Para que a função funcionasse corretamente, utilizamos uma ferramenta da biblioteca chamada "is-eulerian", que verifica se a possibilidade de haver caminho eureliano é real, analisando se todos os vértices possuem grau par.

Implementamos um contador de tempo para a execução de Fleury assim que o usuário escolhe qual dos dois métodos de verificação ele quer utilizar. No terminal, é indicado quanto tempo levou a operação.\vspace{1cm}


\begin{figure}[ht]
\centering
\includegraphics[width=.5\textwidth]{Figure1.png}
\caption{Exemplo de Tempo de execução utilizando Naive}
\label{fig:exampleFig1}
\end{figure}


Com tudo implementado, o novo objetivo foi testar com diferentes números de vertices. As escolhas foram: 100 (cem), 1000 (mil), 10000 (dez mil) e 100000 (cem mil) vertices. Na tabela abaixo, encontram-se os valores de tempo para cada um dos métodos.

Tabela 1. Resultados em um PC com 8GB de RAM

\begin{tabularx}{0.8\textwidth} { 
  | >{\raggedright\arraybackslash}X 
  | >{\centering\arraybackslash}X 
  | >{\raggedleft\arraybackslash}X | }
 \hline
   & Naive & Tarjan \\
 \hline
 100 vértices  & 0.6731 segundos  & 0.1706 segundos  \\
 \hline
 1000 vértices  & 33.28 minutos  & 82.3 segundos  \\
 \hline
 10000 vértices  & 2.27 segundos  & 2.29 segundos \\
 \hline
 100000 vértices  & 21.72 segundos  & 21.73 segundos \\
\hline
\end{tabularx} 


Tabela 2. Resultados em um PC com 16GB de RAM

\begin{tabularx}{0.8\textwidth} { 
  | >{\raggedright\arraybackslash}X 
  | >{\centering\arraybackslash}X 
  | >{\raggedleft\arraybackslash}X | }
 \hline
   & Naive & Tarjan \\
 \hline
 100 vértices  & 0.2893 segundos  & 0.0795 segundos  \\
 \hline
 1000 vértices  & 14.42 minutos  & 42.6 segundos  \\
 \hline
 10000 vértices  & 0.8966 segundos  & 0.9140 segundos \\
 \hline
 100000 vértices  & 9.44 segundos  & 9.47 segundos  \\
\hline
\end{tabularx}


Os grafos de 100 (cem) e 1000 (mil) vértices foram gerados aleatoriamente pelo software gephi, transformados em não-direcionados e eulerizados com o auxílio da biblioteca NetworkX. Já os de 10000 (dez mil) e 100000 (cem mil), são grafos cíclicos, que possuem a quantidade de arestas igual a quantidade de vértices. Optamos por gerar os grafos dessa forma por conta de erros de memória.

Por essa razão, os valores possuem tanta discrepância com os valores de grafos menores, até mesmo entre os dois métodos, praticamente não existe diferença de tempo de execução.

\section{Conclusão}

Após a finalização do trabalho, é notável que o conhecimento da equipe em relação à teoria de grafos foi aumentado. Ademais, foi possível compreender o funcionamento do código fonte de uma biblioteca pronta, a qual foi de grande utilidade e simplificou a manipulação. Foi extremamente necessário possuir tal biblioteca, pois ter a possibilidade de verificar funções básicas, mas essenciais para funções mais complexas foi o diferencial para conseguir realizar essa atividade.

Em relação às descobertas, percebemos que a eficácia do algoritmo de Tarjan é maior, mas é preciso aumentar o limite de recursividade do python, caso contrário, dependendo da quantidade de arestas, surgirão erros. Além disso, foi notável a dificuldida de trabalhar com grafos muito grandes por conta da memória disponível.

\section{Referências}

NETWORKX DEVELOPERS. NetworkX -- Network Analysis in Python. Versão 2.8.8. Novembro, 2022. Disponível em: https://networkx.org/

\end{document}